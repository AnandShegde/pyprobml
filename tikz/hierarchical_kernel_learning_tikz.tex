\documentclass[tikz, border=2pt]{standalone}
% main document, called main.tex
\usepackage{tikz}
\usetikzlibrary{bayesnet}
\usetikzlibrary{shapes}
\begin{document}
% \title{hierarchical_kernel_learning}
% \author{Dhruv Patel}
% \date{July 2022}
\tikzstyle{node} = [ellipse, draw=black!90, fill=black!0, minimum height=5mm, minimum width = 10mm]
\tikzstyle{rectnode1} = [rectangle, draw=blue!60, minimum height=30mm, minimum width = 70mm]
\tikzstyle{rectnode2} = [rectangle, draw=blue!60, minimum height=20mm, minimum width = 60mm]
\tikzstyle{emptynode} = [ellipse, draw=green!0, minimum size = 1mm]
\begin{tikzpicture}
    \node[rectnode1,fill=blue!60, rotate around= {140:(0,0)}] (p) at (-1.8,-2.8) {};
    \node[rectnode2, fill = blue!60, rotate around = {40:(0,0)}] (q) at (1.6, -3.5){};
    \node[node]  (1234) {$1234$};
    \node[emptynode, below = 4mm of 1234](e1){};
    \node[node,  left = 2.2mmof e1](124){$124$};
    \node[node, right = 2.2mmof e1](134){$134$};
    \node[node,  left= 6.3mmof 124](123){$123$};
    \node[node, right  = 5mmof 134](234){$234$};
    \node[node, below  = 5mmof 123](13){$13$};
    \node[node, below  = 5mmof 124](23){$23$};
    \node[node, below  = 5mmof 134](14){$14$};
    \node[node, below  = 5mmof 234](24){$24$};
    \node[node, left = 5mm of 13](12){$12$};
    \node[node, right = 4mm of 24](34){$34$};
    \node[node, below  = 5mmof 13](1){$1$};
    \node[node, below  = 5mmof 23](2){$2$};
    \node[node, below  = 5mmof 14](3){$3$};
    \node[node, below  = 5mmof 24](4){$4$};
    \node[node, below =37mm of 1234](phi){$\phi$};
    \path[thick]
        (123) edge(12) edge(13) edge(23)
        (12) edge(1) edge(2)
        (13) edge(1) edge(3)
        (23) edge(2) edge(3)
        (14) edge(1) edge(4)
        (24) edge(2) edge(4)
        (34) edge(3) edge(4)
        (phi) edge(1) edge(2) edge(3) edge(4);
	
\end{tikzpicture}
\end{document} 
